\chapter{总结和展望}
\section{论文工作总结}
随着近几年移动设备的普及和通信技术的发展,移动网络直播的用户规模呈现爆炸式增长,研究如何优化移动直播的用户体验质量越来越重要。但是无线网络环境较为复杂,移动主播经常和其他人一起竞争带宽,同时由于直播过程中主播的移动等原因,带宽抖动的情况经常会发生。而且移动直播要求用户和主播间有高互动性,端到端的时延最多为几秒,这些因素都为优化用户体验带来了很大的挑战。本文对移动网络环境下主播端存在的性能问题进行研究,提出了一整套的协同解决方案,旨在通过优化主播端的传输性能去优化用户端的体验质量。

首先,本文通过对搭建的demo直播系统的性能测量,发现了移动直播过程中由于网络带宽抖动会导致两种质量问题,分别是短时的带宽抖动带来的应用层丢帧放大效应,以及长时间的带宽抖动带来的上传视频质量降低等问题。为了验证商业直播平台是否也存在类似的问题,我们选取了2个推流端,2个视频服务器组合起来,测量无线网络环境下商业平台的性能。通过实际的测量发现,商业平台也存在以上的两种问题,并不能很好的解决无线网络带来的质量问题。

为了解决上述问题,我们从三个方向进行了探索,涉及了GoP内部优化和GoP整体优化。我们提出的GVBR算法,是一整套提高主播端用户体验质量的协同解决方案,它修改了关键帧间隔的取值,改进了默认的丢帧策略,并且设计了一个有效的码率自适应算法。关键帧间隔的取值综合考虑了视频质量和丢帧现象等两个方面的影响,最后权衡两者给出了关键帧间隔的参考范围。修改后的丢帧策略GreedyDrop考虑了两个或多个GoP同时存在于视频队列中的情况,针对这种情况,GreedyDrop只丢弃旧的GoP的视频帧,保留较新的GoP相关的帧。GreedyDrop算法与离线最优算法Oracle之间的差距很小,只有5\%的差距,而且同时GreedyDrop时间复杂度也很小,为线性时间复杂度。GVBR则根据估计出的带宽和视频队列数据量的差值去选择码率,考虑到视频队列里的剩余数据量可以更精确的计算真实可用的带宽。实验结果表明GVBR相比于最先进的码率自适应算法减少了50\%的丢帧。对比最原始的默认OBS算法,我们提出的一整套协同解决方案,GVBR,将原始的上传失败时长从26秒减少到1秒,改进了95.6\%,同时大幅度改善了用户的体验质量。

\section{未来工作展望}
本文对个人交互直播主播端的用户体验优化研究仍存在不少欠缺的地方,后续的工作可以从下面两个可能的方向来进一步开展:

\textbf{使用新的带宽预测算法提升带宽预测的准确率}。因为本文主要研究的重点是优化主播端的传输性能,所以只是使用了一个现有的带宽预测方案。如果能够进一步提高带宽预测的准确度,相关算法的性能都会更上一个台阶。因此研究一个更加精确的带宽预测方案很有必要,这也是我们下一步工作的重点。

\textbf{将GVBR算法在实际系统中实现}。GVBR算法在设计时就考虑了实际部署的问题,算法的时间复杂度并不高,在实际部署中不会给移动设备带来很大的性能开销。 但实际系统运行的情况远比理论的情况要复杂得多,所以GVBR算法可能在未来应该更结合实际情况,去做一些相应的修改。