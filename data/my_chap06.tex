\chapter{总结和展望}
\section{论文工作总结}
我们提出了GVBR,一整套提高主播端用户体验质量的解决方案,改善默认的丢帧策略,并且设计了一个有效的码率自适应算法。修改后的丢帧策略GreedyDrop考虑了两个或多个GOP同时存在于视频队列中的情况,针对这种情况,只丢弃旧的GOP的所有帧,保留较新的GOP的帧。GreedyDrop与离线最优算法Oracle之间的差距很小,只多了5\%的丢帧。GVBR根据带宽估计和视频队列的数据量的差值去选择码率,考虑到队列里的数据量可以更精确的计算真实可用的带宽。大量的实验表明GVBR相比于最先进的码率自适应算法减少了50\%的丢帧。总之,我们提出的一整套解决方案,GVBR,将原始的上传失败时长从26s减少到1s,改进了96\%。

\section{未来工作展望}
本文对个人交互直播主播端的用户体验优化研究仍有不少欠缺,后续的工作可以从两个方面来进一步开展:

提升带宽预测的准确率。本文因为主要研究的重点是去优化主播端的传输性能,所以只是使用现有的带宽预测方案。但一个更精确的带宽预测算法是必须的。如果能够进一步提高带宽预测的准确度,相关算法的性能都会更上一个台阶。

将GVBR算法在实际系统中实现。GVBR算法在设计时就考虑到了实际部署的问题,算法的时间复杂度并不高,在实际部署中不会给系统带来很大的性能开销。 实际情况比理论的情况要复杂很多,所以GVBR可能还需要一些修改,去更加适应实际情况。