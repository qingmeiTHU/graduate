\thusetup{
  %******************************
  % 注意:
  %   1. 配置里面不要出现空行
  %   2. 不需要的配置信息可以删除
  %******************************
  %
  %=====
  % 秘级
  %=====
  secretlevel={秘密},
  secretyear={10},
  %
  %=========
  % 中文信息
  %=========
  ctitle={移动网络中交互直播主播端的传输优化研究},
  cdegree={工程硕士},
  cdepartment={计算机科学与技术系},
  cmajor={计算机技术},
  cauthor={任青妹},
  csupervisor={崔勇教授},
  %cassosupervisor={陈文光教授}, % 副指导老师
  %ccosupervisor={某某某教授}, % 联合指导老师
  % 日期自动使用当前时间,若需指定按如下方式修改:
  % cdate={超新星纪元},
  %
  % 博士后专有部分
%  cfirstdiscipline={计算机科学与技术},
%  cseconddiscipline={系统结构},
%  postdoctordate={2009年7月——2011年7月},
%  id={编号}, % 可以留空: id={},
%  udc={UDC}, % 可以留空
%  catalognumber={分类号}, % 可以留空
  %
  %=========
  % 英文信息
  %=========
  etitle={Transmission Optimization for Mobile Broadcasters in Personalized Live Video Streaming},
  % 这块比较复杂,需要分情况讨论:
  % 1. 学术型硕士
  %    edegree:必须为Master of Arts或Master of Science(注意大小写)
  %             “哲学、文学、历史学、法学、教育学、艺术学门类,公共管理学科
  %              填写Master of Arts,其它填写Master of Science”
  %    emajor:“获得一级学科授权的学科填写一级学科名称,其它填写二级学科名称”
  % 2. 专业型硕士
  %    edegree:“填写专业学位英文名称全称”
  %    emajor:“工程硕士填写工程领域,其它专业学位不填写此项”
  % 3. 学术型博士
  %    edegree:Doctor of Philosophy(注意大小写)
  %    emajor:“获得一级学科授权的学科填写一级学科名称,其它填写二级学科名称”
  % 4. 专业型博士
  %    edegree:“填写专业学位英文名称全称”
  %    emajor:不填写此项
  edegree={Master of Engineering},
  emajor={Computer Technology},
  eauthor={Ren Qingmei},
  esupervisor={Professor Cui Yong},
  % 日期自动生成,若需指定按如下方式修改:
  % edate={December, 2005}
  %
  % 关键词用“英文逗号”分割
  ckeywords={交互直播, 主播端, 用户体验质量, 移动网络,自适应码率},
  ekeywords={Personalized Live Video Streaming, Broadcaster, QoE, Mobile Network, Adaptive Bitrate}
}

% 定义中英文摘要和关键字
\begin{cabstract}
移动设备的大量普及以及LTE等通信技术的迅速发展让用户不再满足于单纯的观看视频直播,用户逐渐成为直播内容的贡献者。传统的流媒体直播技术,不能有效地应对无线网络的带宽抖动,也不能满足交互直播端到端时延短以及高交互的性能要求。RTMP协议因为其流式传输的特性以及较低的端到端时延在移动直播中得到了广泛的应用。现有流媒体直播研究侧重于对直播系统架构和用户直播行为规律性的研究,关于优化视频直播传输质量的研究较为稀少。随着移动直播用户量的爆炸式增长,优化直播传输的质量迫在眉睫。

主播端的用户服务质量在交互直播服务中起着至关重要的作用,主播端上传过程中发生的任何延迟和失败都会对所有观看用户产生影响。本文通过测量多个流行商业直播平台的主播端,发现当无线网络的带宽发生抖动时,所有平台的主播端都会观测到长时间的丢帧现象,严重损害了观众的用户体验质量。

通过分析开源直播软件的源码发现,丢帧效应产生的主要原因有两点,视频编码带来的帧间依赖问题,以及网络状况不好时视频数据产生速率高于网络容量。为了减少丢帧现象的发生,本文从关键帧间隔,视频帧的丢帧策略和码率选择三个角度分别着手。网络带宽已知条件下最大化数据传输量的模型,从最优化的角度给出了离线最优的丢帧策略,保证了视频帧的解码约束以及带宽容量约束。码率选择方面,本文尝试固定丢帧策略,综合考虑实时码率以及码率切换,丢帧带来的质量损失,去最大化用户体验质量的收益。

为了优化主播端的用户体验质量,我们协同上述三个策略设计了一整套的解决方案:(1) 最优的关键帧间隔选择策略。关键帧间隔选择时综合考虑视频帧间的依赖关系以及视频压缩比,在丢帧效应和视频质量之间寻求一个平衡点;
(2) 简单且有效的智能丢帧策略。当视频帧队列溢出时选择性的丢弃旧的GoP,用来缓解商业直播平台应用层丢帧的放大效应;
(3) GoP粒度的码率自适应策略。基于带宽估计值和视频队列数据量之间的差值来选择合适的码率,通过调节系数为队列缓存小的主播端定制化策略。

本文选取了三个点播领域最新的码率自适应算法作为对比算法,在LTE和Wifi网络环境下分别进行了仿真。实验结果表明,本文提出的解决方案在维持相同码率的条件下,大幅度减少了用户的上传失败问题。
\end{cabstract}

% 如果习惯关键字跟在摘要文字后面,可以用直接命令来设置,如下:
% \ckeywords{\TeX, \LaTeX, CJK, 模板, 论文}

\begin{eabstract}
With the popularization of mobile devices and the rapid development of communication technologies, instead of simply watching live video, users more and more participate in contributing the live streaming. Traditional streaming technology cannot effectively cope with the bandwidth jitter of the wireless network, nor can it meet the requirements of low-latency and high interactive live streaming. Due to the low end-to-end delay, RTMP is widely used in mobile live streaming. Recent researches focus on the system architecture of live streaming and the pattern of users' behavior, studies on optimizing the QoE of live video transmission is rare to find. With a huge number of users, optimizing the QoE of live video is imminent.

Ensuring high video quality of experience (QoE) on the
broadcaster side is critical for interactive live streaming
services, because any delay on the broadcaster side can cause
negative impact on all viewers. Through measurements on
multiple popular live video streaming platforms, we find that
they all suffer from broadcaster-side video quality degradation
caused by unnecessarily persistent video interruptions
in the presence of transient bandwidth fluctuations. 

Analyzing the source code of the open source live broadcast software, there are two main reasons for the frame dropping issue, the inter-frame dependency caused by video encoding, and that the video data generation rate is higher than the network capacity when the network is in bad condition. In order to reduce the frame dropping, this paper tries three aspects: key frame interval, frame dropping strategy and video bitrate selection. Assume known the network bandwidth, we try ro maximize the transmitted data under the constraints of frame decodable and the bandwidth capacity, which provides an offline optimal frame dropping strategy. As for the video bitrate, we take the bitrate, rate switch penalty and frame dropping into consideration, use the known frame dropping strategy to maximize the QoE utility.

This paper takes a holistic stance, and presents a suite of
solutions that optimizes the broadcaster-side QoE through (1)
a keyframe placement strategy that dynamically trades crossframe
compression for lowered inter-frame interdependency,
(2) a simple-yet-efficient frame dropping strategy to prevent
excessive frame drops observed in many popular streaming
platforms, and (3) finally, a RTMP-based bitrate adaptation
strategy customized for video broadcasters who have extremely
shallow buffer (below one second). We compare our solution with several state-of-the-art video adaptation algorithms in a variety of network conditions, and find
that our solution can substantially reduce the frame drops, while achieving comparable
bitrate.
\end{eabstract}

% \ekeywords{\TeX, \LaTeX, CJK, template, thesis}
