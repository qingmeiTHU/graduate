\begin{resume}

  \resumeitem{个人简历}

  1992 年 8 月 16 日出生于 安徽 省 淮北市 。

  2011 年 9 月考入 北京邮电大学 信息与通信工程学院 系 通信工程 专业,2015 年 7 月本科毕业并获得 工学 学士学位。

  2015 年 9 月免试进入 清华 大学 计算机科学与技术 系攻读 硕士 学位至今。

  \researchitem{发表的学术论文} % 发表的和录用的合在一起

  % 1. 已经刊载的学术论文(本人是第一作者,或者导师为第一作者本人是第二作者)
%  \begin{publications}
%    \item 杨轶, 张宁欣, 任天令, 等. 硅基铁电微声学器件中薄膜残余应力的研究. 中国机
%      械工程, 2005, 16(14):1289-1291. (EI 收录, 检索号:0534931 2907.)
%    \item 杨轶, 张宁欣, 任天令, 等. 集成铁电器件中的关键工艺研究. 仪器仪表学报,
%      2003, 24(S4):192-193. (EI 源刊.)
%  \end{publications}
%
%  % 2. 尚未刊载,但已经接到正式录用函的学术论文(本人为第一作者,或者
%  %    导师为第一作者本人是第二作者)。
%  \begin{publications}[before=\publicationskip,after=\publicationskip]
%    \item Yang Y, Ren T L, Zhu Y P, et al. PMUTs for handwriting recognition. In
%      press. (已被 Integrated Ferroelectrics 录用. SCI 源刊.)
%  \end{publications}

  % 3. 其他学术论文。可列出除上述两种情况以外的其他学术论文,但必须是
  %    已经刊载或者收到正式录用函的论文。
  \begin{publications}
    \item Qingmei Ren, Yong Cui, Wenfei Wu, Changfeng Chen, Yuchi Chen, Jiangchuan Liu and Hongyi Huang. Improving Quality of Experience for Mobile Broadcasters in Personalized Live Video Streaming. Accepted by IEEE/ACM International Symposium on Quality of Service (IWQoS) 2018 Short Paper.
    \item Yong Cui, Jian Song, Kui Ren, Minming Li, Zongpeng Li, Qingmei Ren and Yangjun Zhang. Software defined cooperative offloading for mobile cloudlets[J]. IEEE/ACM Transactions on Networking (TON), 2017.
    \item Yong Cui, Jian Song, Minming Li, Qingmei Ren, Yangjun Zhang and Xuejun Cai. SDN-based big data caching in ISP networks[J]. IEEE Transactions on Big Data (TBD), 2017.
  \end{publications}

  \researchitem{研究成果} % 有就写,没有就删除
  \begin{achievements}
    \item 崔勇, 宋健, 任青妹. 智能终端能耗优化的自适应流媒体分发方法: 中国, CN105245919B. (中国专利公告号.)
  \end{achievements}

\end{resume}
