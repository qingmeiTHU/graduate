\begin{resume}

  \resumeitem{个人简历}

  1992 年 8 月 16 日出生于 安徽 省 淮北市 。

  2011 年 9 月考入 北京邮电大学 大学 信息与通信工程学院 系 通信工程 专业,2015 年 7 月本科毕业并获得 工学 学士学位。

  2015 年 9 月免试进入 清华 大学 计算机科学与技术 系攻读 硕士 学位至今。

  \researchitem{发表的学术论文} % 发表的和录用的合在一起

  % 1. 已经刊载的学术论文(本人是第一作者,或者导师为第一作者本人是第二作者)
  \begin{publications}
    \item Yang Y, Ren T L, Zhang L T, et al. Miniature microphone with silicon-
      based ferroelectric thin films. Integrated Ferroelectrics, 2003,
      52:229-235. (SCI 收录, 检索号:758FZ.)
    \item 杨轶, 张宁欣, 任天令, 等. 硅基铁电微声学器件中薄膜残余应力的研究. 中国机
      械工程, 2005, 16(14):1289-1291. (EI 收录, 检索号:0534931 2907.)
    \item 杨轶, 张宁欣, 任天令, 等. 集成铁电器件中的关键工艺研究. 仪器仪表学报,
      2003, 24(S4):192-193. (EI 源刊.)
  \end{publications}

  % 2. 尚未刊载,但已经接到正式录用函的学术论文(本人为第一作者,或者
  %    导师为第一作者本人是第二作者)。
  \begin{publications}[before=\publicationskip,after=\publicationskip]
    \item Yang Y, Ren T L, Zhu Y P, et al. PMUTs for handwriting recognition. In
      press. (已被 Integrated Ferroelectrics 录用. SCI 源刊.)
  \end{publications}

  % 3. 其他学术论文。可列出除上述两种情况以外的其他学术论文,但必须是
  %    已经刊载或者收到正式录用函的论文。
  \begin{publications}
    \item Cui Y, Song J, Ren K, et al. Software defined cooperative offloading for mobile cloudlets[J]. IEEE/ACM Transactions on Networking, 2017, 25(3): 1746-1760. (SCI 收录, 检索号
      :896KM)
    \item Cui Y, Song J, Li M, et al. SDN-based big data caching in ISP networks[J]. IEEE Transactions on Big Data, 2017. (EI 收录, 检索号:06129773469)
  \end{publications}

  \researchitem{研究成果} % 有就写,没有就删除
  \begin{achievements}
    \item 崔勇, 宋健, 任青妹. 智能终端能耗优化的自适应流媒体分发方法: 中国, CN105245919B. (中国专利公布号.)
  \end{achievements}

\end{resume}
