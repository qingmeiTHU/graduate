\chapter{绪论}
\label{cha:intro}

\section{课题研究背景}
近年来个人直播呈爆炸式增长。根据中国互联网络信息中心(CNNIC)统计,截至2016年12月,网络直播用户达到3.44亿,占总体网民的47.1\%,相比2016年6月增长了1932万。随着移动设备的发展,从2016年下半年开始,移动直播用户赶超PC端用户。个人直播应用繁荣发展,斗鱼和熊猫有强大的粉丝群体,用户可以在Facebook Live和Periscope上分享生活,在Twitch上分享游戏视频。

多数关于个人交互直播的研究都着眼于直播用户行为的分析和系统架构,但很少有研究专注于优化主播端的上传质量。但对于直播应用来说,主播端的视频传输质量尤为重要。主播端的任何延迟或者上传失败都会反应到所有用户,主播影响着全部的用户。除此之外,上传的视频质量为所有用户设置了一个上限。因此,很多情况下,主播通常只上传最高码率的视频。

个人直播主播端的传输优化不仅很重要,而且不同于传统的直播应用,比如,ESPN和CCTV的一些赛事直播。传统的直播应用都会提前分配好带宽,传输时延一般在几十秒左右。但在个人交互直播中,主播可能多数使用无线网络,同时移动直播,带来的网络抖动会比较剧烈;另外,端到端的时延必须在几秒以下,才能满足主播和观众交互的要求,比如,提问或者送礼物。个人交互直播新的挑战可以总结为一句话:在变化的带宽环境下提供高传输质量,包括视频质量和延时要求。
\section{主要研究目标和贡献}
将传统的直播技术直接运用到个人交互直播上会在主播端带来许多问题。我们通过测量流行的直播应用,发现这些直播平台都存在两个普遍的问题:
\begin{enumerate}
  \item 短暂的网络抖动会带来一个放大效应,导致长时间的视频质量下降。比如,主播端1s的网络抖动会造成数秒的视频黑屏或暂停。
  \item 这些直播应用都无法有效应对长时间的带宽降低。由于小区切换,WiFi蜂窝网切换或者设备移动等原因,主播端经常会有长时间持久的带宽波动,无法有效应对会导致严重的视频质量下降。
\end{enumerate}

放大效应是由于RTMP的丢包行为导致。当视频的队列溢出时,主播端会主动丢包,会导致丢失很多报文,因此用户端会观测到长时间的视频卡顿。而且,一些直接的简单放大,比如,增大队列长度,其他的丢包方式,会不满足至少一个质量要求,或者增大端到端的时延,或者降低视频的质量。比如,增大主播端的视频队列长度,可以很好的解决短暂的带宽下降,但端到端的最高时延也会相应的增大。

另外,对于长时间的网络带宽降低,现有的方法都集中在研究用户端的自适应码率,视频的队列长度一般维持在10s左右。这些在点播领域效果比较好的自适应码率方案,在主播端表现不好,因为主播端的视频队列通常只有1s左右。

本文中,我们提出了一套解决方案,简称为GVBR(Greedy Variable Bitrate Solution),大大提高了交互直播中主播端的视频质量。我们的主要出发点是主播端的质量问题可以通过跨层协同设计来解决。主要包括三层,RTMP层,帧级别,以及GoP级别,RTMP层的配置调整主要包括,关键帧间隔,视频队列大小,帧级别的解决方案主要是丢帧策略,三个级别的优化目标都是视频的质量和及时性。虽然我们的整套解决方案修改了视频流协议的两层,但所有的改动都不用修改内部逻辑,或者更改可调的参数(例如,关键帧间隔),或者改变软件中的控制逻辑(丢帧的逻辑和码率自适应策略)。

我们的初步仿真说明,一套好的RTMP层协议可以大大的提高视频质量。通过在不同网络环境下的大规模仿真,我们发现GVBR同目前现先进的算法比,可以减少至少50\%的丢帧,相比原始的算法,在保证同样视频码率的情况下,视频卡顿的概率减少90。

总而言之,本文主要有两个贡献点:
\begin{itemize}
  \item 我们是第一个把目光着眼于主播端视频质量的人。通过对于流行视频直播平台的测量,我们发现一个普遍的主播端质量问题,短暂的带宽波动会播放带来的长时间视频卡顿。
  \item 我们提出了一整套解决方案,去协同解决视频质量问题,包括帧编码和帧优先级策略,以及GoP级别的比特率自适应策略,。
\end{itemize}

\section{文章组织架构}
本文的内容共有六章,按照如下方式展开:

第一章是绪论,介绍课题研背景和我们的主要研究内容。

第二章介绍个人直播相关的一些背景知识,主要包括个人直播的架构、特点、以及性能要求。

第三章从我们搭建的平台出发,给出测量中发现的问题,目前的直播平台并不能很好的满足性能要求;并给出在商业平台中的测量,实验发现商业平台也不能很好的解决上述问题。

第四章,剖析直播软件的源码,发现上述问题出现的原因,主要是由于视频编码输出的码率不能实时的按照带宽变化,最后我们给出了一些优化设计的原则。

第五章根据设计准则,我们设计了一整套的解决方案,涵盖帧级到GoP级。

第六章与之前的算法进行比较,进行充分的实验仿真。

第七章总结现有的工作,并展望未来的研究方向。

